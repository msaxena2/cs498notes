\documentclass{beamer}
\usetheme{CambridgeUS}
\usecolortheme{dolphin}
\usepackage[utf8]{inputenc}
\usepackage{dirtytalk,amsmath,tikz,mathtools,amsmath, empheq}
\usepackage{fixltx2e}
\usepackage{resizegather}
\usetikzlibrary{automata, positioning}
\graphicspath{ {./images/} }
\tikzset{->, initial text=$$}
\DeclareMathAlphabet{\mathpzc}{OT1}{pzc}{m}{it}

%Information to be included in the title page:
\title[Parameter Invariant Monitoring for STL]{Parameter Invariant Monitoring for Signal Temporal Logic}
\subtitle{Nima Roohi, Ramneet Kaur, James Weimer, Oleg Sokolsky, Insup Lee}
\author{Manasvi Saxena}

%\setbeameroption{hide notes} % Only slides
%\setbeameroption{show only notes} % Only notes
%\setbeameroption{show notes on second screen=right} % Both
%\setbeamertemplate{note page}{\pagecolor{yellow!5}\insertnote}\usepackage{palatino}
\setbeamerfont{caption}{size=\scriptsize}
\date{}


\newcommand{\always}[1]{\square_{#1}}
\newcommand{\eventually}[1]{\lozenge_{#1}}

\newcommand{\typeTime}{\mathbb{R}_{\geq 0}}
\newcommand{\typeReal}{\mathbb{R}}

% text versions
\newcommand{\Z}{\text{Z}}
\newcommand{\f}{\text{f}}
\newcommand{\varT}{\text{t}}

\begin{document}

\frame{\titlepage}

\section{Introduction}
\subsection{Motivation}

\begin{frame}
    \frametitle{Motivations}

    Challenges in monitoring Real Time Systems -
    \begin{itemize}
        \item \textit{Partially Observable} States.
        \item \textit{Partially Observable} Traces.
        \item \textit{High Computation} Cost.
    \end{itemize}

    \pause

    Extends Parameter Invariant (PAIN) tests to -
    \begin{itemize}
        \item STL to support continuous systems
        \item Efficiently monitor STL online.
    \end{itemize}

\end{frame}

\begin{frame}
    \frametitle{Example}
    \framesubtitle{Monitoring a Diabetic Patient}

    STL formula, \say{If patient has a meal (M), then (s)he
    either recieved bolus $t_1$ units ago, or will recieve one in $t_2$ units}
        $$ \always{ [ t_1, \infty )}
    \left( \left (M > c_1 \right ) \to \eventually{\left(-t_1, t_2 \right)}
    \left( B > c_2 \right ) \right)$$.
    \pause
    \begin{itemize}
        \item \textit{State} variables not directly Observable.
        \item \textit{Output} variables affected by sensor and environment noise.
        \item Models are \textit{parametric}. Estimating (\textit{nuissance}) parameters is not
            feasible.
    \end{itemize}
\end{frame}
\begin{frame}
    \frametitle{Example}
    \framesubtitle{Monitoring a Diabetic Patient}

    STL formula, \say{If patient has a meal (M), then (s)he
    either recieved bolus $t_1$ units ago, or will recieve one in $t_2$ units}
    \begin{itemize}
        \item B (Bolus) is observable, but M (Meal) is not. Furthermore, B
            can be affected by noise.
        \pause
        \item Avoid relying on exact models of the system.
        \item Rely on \textit{probabilistic beliefs} about
            system state at a given point in time.
        \pause
        \item Furthermore, use \textit{maximally invariant} statistics
            to derive \textit{probabilistic beliefs}
    \end{itemize}
\end{frame}

\section{Preliminaries}
\subsection{Signal Temporal Logic (STL)}
\frametitle{Notation}
\begin{frame}
    \begin{block}{Signals}
        Given a finite set of variables $\Z$, signal
        $f : \typeTime \to \typeReal^{\Z}$ is a mapping
        from a point in \textit{time} to a \textit{valuation} of $\Z$.
    \end{block}
    \pause
    \begin{block}{Formulas over Signals}
        \begin{itemize}
            \item An formula $\theta : \typeReal^{\Z} \to \typeReal$
                  maps a \textit{valuation} to a real value.
            \item Function $\theta \circ f : \typeTime \to \typeReal$
                  maps points in time to values defined by $\theta$.
            \item At time $t \in \typeTime$, signal $f$ is \textit{true}
                w.r.t. $\theta$ whenever $\theta(f(t)) > 0 $.
                Conversely, $f$ is \textit{false} w.r.t $\theta$
                w.r.t. $\theta$ whenever $\theta(f(t)) < 0 $.
            \item Whenever, $\theta(f (t)) = 0$, value of $f$
                is \textit{unknown} on $\theta$ at time $t$.
            \item $\theta(f(t))$, is the \textit{robustness degree}.
                Larger \textit{robustness degree} values signify
                greater \textit{belief} about signals' adherence
                to a formula.
        \end{itemize}
    \end{block}
\end{frame}

\end{document}



