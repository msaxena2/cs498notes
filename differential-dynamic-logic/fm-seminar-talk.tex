\documentclass{beamer}
\usecolortheme{whale}
\usepackage[utf8]{inputenc}
\usepackage{dirtytalk,graphicx,amsmath,tikz}
\usetikzlibrary{automata, positioning}
\graphicspath{ {./images/} }
\tikzset{->, initial text=$$}
%Information to be included in the title page:
\title{Logic and Hybrid Systems}
\author{Manasvi Saxena}
\institute{Formal Systems Lab, UIUC}
\date{}

\setbeameroption{hide notes} % Only slides
%\setbeameroption{show only notes} % Only notes
%\setbeameroption{show notes on second screen=right} % Both
\setbeamertemplate{note page}{\pagecolor{yellow!5}\insertnote}\usepackage{palatino}
\setbeamerfont{caption}{size=\scriptsize}

\newcommand{\dL}{\text{\upshape\textsf{d{\kern-0.05em}L}}}

\begin{document}

\frame{\titlepage}

\begin{frame}
\frametitle{Hybrid Systems}

\begin{itemize}
  \item Dynamical Systems exhibiting both discrete (jump) and continuous (flow) behaviors.
  \item Serve as models of physical systems, from thermostats to trains.
  \item Continuous dynamics specified using Differential Equations.
\end{itemize}

\end{frame}

\begin{frame}
  \frametitle{Differential Dynamic Logic (\dL)}
  \begin{itemize}
    \item Main focus - Differential Dynamic Logic for Hybrid Systems (Andre
      Platzer).
      \pause
    \item Practical deductive verification of hybrid systems.
      \pause
    \item Introduces Hybrid Program - program notation for hybrid systems.
      \pause
    \item Dynamic Logic for Hybrid Programs, a generalization of Dynamic Logic.
      \pause
    \item Suited for automation.
  \end{itemize}

\end{frame}

\begin{frame}
\frametitle{Hybrid Automata}
\begin{itemize}
  \item Commonly used to model Hybrid Systems, via Graphs.
  \item Nodes specify continuous dynamics. Edges describe discrete transitions.
  \item Intuitive, but not suitable for deductive verification.
\end{itemize}
    \pause
\begin{figure}\label{fig:train-HA}
  \centering
  
  \node[state,initial, scale=0.70] (accel)   {%
  $\begin{aligned}
      \textit{accel} \\
      z' = v \\
      v' = a \\
      v \leq 11
  \end{aligned}$
  };

  \node[state,scale=0.70] (brake) [right of = accel] {%
  $\begin{aligned}
      \textit{brake} \\
      z' = v \\
      v' = a \\
      v \geq 0
  \end{aligned}$
   };

   \draw (accel) edge[bend left, above] node[midway,below,scale=0.7] {$a := -B$}
   node[midway,above,scale=0.7] {$v \geq 10$} (brake);
   \draw (brake) edge[bend left, below] node[midway,below,scale=0.7] {$a := a+5$}
   node[midway,above,scale=0.7] {$v \leq 1$} (accel);

  \caption{Hybrid Automata (simplified) of a Train Control System}
\end{figure}
\end{frame}

\begin{frame}{Differential Dynamic Logic}{Motivations}
  \begin{itemize}
    \item \textbf{First Order Logic} - No builtin means for referring to state
      transitions.
    \item \textbf{Temporal Logics} - Modal operators allow referring to state transitions.
      But valid formulas only express generic facts.
      \pause
    \item \textbf{Dynamic Logic (DL)} - Combines operational system models with
      operators for reasoning.
      \begin{itemize}
        \item Provides parameterized modal operators, $[\alpha]$,
          $\langle\alpha\rangle$
          that refer to states reachable by system $\alpha$.
        \item $[\alpha]\phi$ expresses all states reachable by $\alpha$ satisfy
          $\phi$, allowing reasoning about discrete systems.
        \item Say $(b > 0) \to [a := -b] (a < 0) $ expresses a
          discrete transition. Using DL's calculus, we get $(b > 0) \vdash (a < 0)
          [ b / a ]$. Convenient for  reasoning about discrete behavior.
        \pause
        \item No built in notion for describing or reasoning about continuous dynamics.
      \end{itemize}
  \end{itemize}
\end{frame}

\begin{frame}{Differential Dynamic Logic}{Motivations}
  \begin{itemize}
    \item Generalize DL so operational models $\alpha$ can be
      used in modal formulas like $[\alpha]\phi$. $\dL$ refers to
      generalized models as \say{Hybrid Programs}.
    \item A compositional calculus for verification. Decompose
      $[\alpha]\phi$ into an equivalent formula $[\alpha_1]\phi_1 \wedge
      [\alpha_2]\phi_2$.
    \item Prove subsystems and subproperties $[\alpha_i]\phi_i$ independently
      and combine results conjuntively.
    \item Complete relative to handling of differential equations.
  \end{itemize}
\end{frame}

\begin{frame}{Differential Dynamic Logic}{Syntax and Semantics}
  \begin{description}
    \item{} $\dL$ formulas built over $V$, set of real-valued logical variables and
     signature $\Sigma$ containing functions, predicate symbols over reals, like
     $0, 1, +, \geq$.
  \item {}$\Sigma$ also contains \textit{System State Variables}. Unlike rigid
   symbols, like $1, 2$, their interpretation can change from state to state.
  \end{description}
\end{frame}

\end{document}

