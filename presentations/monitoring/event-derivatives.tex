\documentclass[11pt]{article}
\usepackage{graphicx}
\usepackage{hyperref}
\usepackage{dirtytalk}
\usepackage{mathbbol}
\usepackage{xfrac}
\usepackage{enumitem}
\usepackage{amsmath}
\usepackage{amssymb}
\usepackage{mathtools}
\usepackage{hyperref}

\newcommand{\derivative}[2]{\delta_{#1} #2}
\newcommand{\until}{\ \mathcal{U}\ }
\newcommand{\Prop}{\mathcal{P}}
\newcommand{\State}{\textit{State}}
\newcommand{\Transition}{\textit{Transition}}
\newcommand{\unsafe}{\textit{unsafe}}

\begin{document}

\title{Monitoring via Rewriting}

\section{Formula Based Monitoring}

\subsection{Problem Statement}
Given a finite trace $\pi = e_1,e_2,\ldots$, and an propositional LTL formula
$\varphi$, we need an algorithm for checking satisfaction that uses the formula
itself as the \say{state} of the monitor.

\subsection{Derivatives}
We take derivatives of the formula of interest to derive a new formula that the
remainder of the trace must satisfy. Formally,  $e\tau \vdash \varphi$ iff $\tau
\vdash \derivative{e}{\varphi}$. For propositional LTL, derivatives are defined
as follows -
\begin{equation}
\begin{split}
  \derivative{e}{\circ \varphi} &= \varphi \\
  \derivative{e}{\square \varphi} &= \derivative{e}{\varphi} \wedge (\square \varphi) \\
  \derivative{e}{\lozenge \varphi} &= \derivative{e}{\varphi} \vee (\lozenge \varphi) \\
  \derivative{e}{\varphi \until \psi} &= \derivative{e}{\varphi} \vee
  \derivative{e}{\psi} \wedge \varphi \until \psi
\end{split}
\end{equation}

For propositional LTL, where an event is a set of true propositions, we further
have

\begin{equation}
  \begin{split}
  \derivative{ \{ p \} }{ \Prop } &=
    \begin{cases}
      \top & \; \text{if}\, p = \Prop \\
      \bot & \; \text{otherwise}
    \end{cases} \\
    \derivative{ \{ p \} \cup \textit{R} }{ \Prop } &=
    \begin{cases}
      \top & \; \text{if}\, p = \Prop \\
      \derivative{ \textit{R} }{\Prop} & \; \text{otherwise}
    \end{cases} \\
  \end{split}
\end{equation}

So for example, if we have $\pi = b , a b , a^*$, and $\varphi = \square (\lozenge a ) $
then the monitoring algorithm world work as -
\begin{align*}
  &\null b, a b , a^*              &\null \vdash \; &\null \square (\lozenge a ) \\
  \leftrightarrow &\null \; a b, a^* &\null \vdash \; &\null \derivative{b}{\lozenge a} \wedge (\square \lozenge a)  \\
  \leftrightarrow &\null \; a b, a^* &\null \vdash \; &\null (\bot \vee
  \lozenge(a) = \lozenge(a))  \wedge (\square \lozenge a)  \\
  \leftrightarrow &\null \;  a^* &\null \vdash \; &\null \derivative{a b }{\lozenge a} \wedge \derivative{a b}{\lozenge a} \wedge  (\square \lozenge a)  \\
  \leftrightarrow &\null \;  a^* &\null \vdash \; &\null \derivative{a b }{a}  \wedge  (\square \lozenge a)  \\
  \leftrightarrow &\null \;  a^* &\null \vdash \; &\null \top  \wedge  (\square \lozenge a)  \\
  \leftrightarrow &\null \; a^* &\null \vdash \; &\null \derivative{a^*}{\lozenge a} \wedge
   (\square \lozenge a) \\
  &\null = \top
\end{align*}

\subsection{Generalizing to ML formulas over Transition Systems}
Consider a theory $\mathbb{\Sigma}^{\text{TS}} =
(\{\State,\Transition\}, \bullet \in \Sigma_{\Transition \, \State, \State})$.

Given sort $\textit{State}$, an event is a concrete value
$e \in M_{\State}$, where $M_{\State}$ is the carrier set of sort $\State$.

Using reachability, safety can be
expressed as
$ \psi_\textit{safety} = \varphi_{\textit{init}} \to \lozenge \varphi_{\unsafe}$.

\begin{align*}
  \derivative{e}{\varphi} &= \lceil e \to \varphi \rceil  \\
\end{align*}

\end{document}
